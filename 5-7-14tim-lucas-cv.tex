\documentclass[a4paper,10pt,reqno,oneside]{amsart}
\usepackage[small]{caption}
\usepackage[usenames,dvipsnames]{color}
\usepackage[colorlinks=TRUE,linkcolor=black,urlcolor=blue,pagebackref=TRUE,]{hyperref}
\usepackage{overcite,amsfonts,fancyhdr,graphicx,lastpage,rotating,multirow,subfig,fixltx2e, stfloats,lscape,lettrine,txfonts,palatino,url,geometry,xcolor,multicol,setspace,hanging,setspace,titlesec}

\usepackage[T1]{fontenc}
%\usepackage[scaled]{berasans}
\usepackage{lmodern}

\usepackage{helvet}
%\renewcommand*\familydefault{\sfdefault} 
\setlength\parindent{0in}



%If i want these back in i need to use package titlesec
\titleformat{\section}[hang]{\raggedright \bf \color{blue}}{\thesection}{}{}[]
\titlespacing*{\section}{0pt}{12pt}{5pt}

\titleformat{\subsection}[hang]{\raggedright \it \small}{\thesubsection}{}{}[]
\titlespacing*{\subsection}{0pt}{5pt}{0pt}

\renewcommand{\labelitemii}{$-$}


\geometry{a4paper,tmargin=10mm,bmargin=15mm,lmargin=24mm,rmargin=27mm}


\begin{document}

\title{}
\author{}
\date{}

\maketitle

\thispagestyle{plain}

\begin{flushleft}
\vspace{-2.2cm}
\Large{\emph{Curriculum Vitae}}\\
\fontsize{6mm}{1cm}
\selectfont
\Huge{\textsf{Tim C.D. Lucas}}
\end{flushleft}
\normalfont

\section*{Contact}



Phone: 07415863536\\
Email: \href{mailto:timcdlucas@gmail.com}{timcdlucas@gmail.com}\\


 
\vspace{7mm}
\section*{Education}

\emph{October 2012--Present} \\
\textbf{PhD} \emph{Social structure and network epidemiology in bat zoonoses}, University College London \small{(UCL)}\\


\small{CoMPLEX}, Genetics, Evolution and Environment and Statistical Science\\
Kate Jones and Hilde Wilkinson Herbots\\

I am using complex network theory to study the epidemiology of bat-borne diseases. As bats carry a number of important zoonotic diseases, understanding the spread of these diseases within the bat population and how this affects spillover to humans and livestock is increasingly important. The unusually social nature of bat populations will strongly affect how diseases spread.\\
 

\emph{2011--2012} --- \textbf{MRes} Modelling biological complexity, \small{CoMPLEX, UCL}: \textbf{Merit}\\


\emph{2006--2010} --- \textbf{MBioSci} Zoology, University of Sheffield: \textbf{First Class}\\

\section*{Computer Skills}
\begin{itemize}
\item Experienced with R (6 years), Java, Matlab, Mathematica and others.
\item Comfortable with Windows, Mac or Linux.
\item LaTeX, webdesign etc.
\end{itemize}

\section*{Projects and other experience}

Summer 2012 --- \small{CoMPLEX} summer project\\
\emph{A gas model for estimating bat abundances using acoustic data.}\\
I adapted an analytical model for camera trap studies to be applicable to acoustic data. I applied the model using R to a pan-European bat survey. \\

May 2012 --- \small{CoMPLEX} case presentation\\
\emph{Moment closures for pair approximations in spatial biology.}\\
I compared a number of moment closures for a pair-approximated logistic model of tree population growth to lattice simulations written in Mathematica. \\

December 2011 --- \small{CoMPLEX} case presentation\\
\emph{Gaussian process machine learning for bat identification.}\\
I applied a novel machine learning method to a library of bat calls in matlab. I compared the effectiveness of this method to standard machine learning methods applied in R. \\

August--July 2011\\
Two months fieldwork in Panam\'{a} on two projects: studying \emph{Anolis} dewlap evolution with Dr.\ Jessica Stapley and studying gut length plasticity in Red-eyed tree frogs with the Smithsonian Tropical Research Institute. \\

May 2011\\
Two weeks volunteer field work trapping Darwin's foxes in Chil\'{e} with a charity Chilo\'{e} Silvestre.\\

July--October 2010\\
I studied the evolutionary response of plant communities to climate change by collecting, propagating and analysing plants collected from the field. University of Sheffield summer studentship. \\

2009--2010 --- \small{MBioSci} project\\
\emph{Examining drivers of malaria dynamics in Thailand using wavelet analysis.}\\
This project involved analysing a large dataset of Thai malaria cases using wavelet analysis in R. \\

June--August 2009\\
I studied collective foraging behaviour by programming a complex 3D foraging simulation in java. York Centre for Complex Systems Analysis, University of York summer studentship.\\

2008--2009 --- \small{BSci} project\\
\emph{A Theoretical assessment of the delyfing fitness measure.}\\
I applied the delyfing fitness metric to simulated evolving populations. Simulations were written in R. \\




\end{document}

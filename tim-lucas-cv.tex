%%%%%%%%%%%%%%%%%%%%%%%%%%%%%%%%%%%%%%%%%
% Classicthesis-Styled CV
% LaTeX Template
% Version 1.0 (22/2/13)
%
% This template has been downloaded from:
% http://www.LaTeXTemplates.com
%
% Original author:
% Alessandro Plasmati
%
% License:
% CC BY-NC-SA 3.0 (http://creativecommons.org/licenses/by-nc-sa/3.0/)
%
%%%%%%%%%%%%%%%%%%%%%%%%%%%%%%%%%%%%%%%%%

%----------------------------------------------------------------------------------------
%	PACKAGES AND OTHER DOCUMENT CONFIGURATIONS
%----------------------------------------------------------------------------------------

\documentclass{scrartcl}

\reversemarginpar % Move the margin to the left of the page 

\newcommand{\MarginText}[1]{\marginpar{\raggedleft\itshape\small#1}} % New command defining the margin text style

\usepackage[nochapters]{classicthesis} % Use the classicthesis style for the style of the document
\usepackage[LabelsAligned]{currvita} % Use the currvita style for the layout of the document
\usepackage{graphicx}

\usepackage{multicol}


\renewcommand{\cvheadingfont}{\LARGE\color{Maroon}} % Font color of your name at the top

\newcommand{\cemph}{\textit\color{Maroon}}

\usepackage{hyperref} % Required for adding links	and customizing them
\hypersetup{colorlinks, breaklinks, urlcolor=Maroon, linkcolor=Maroon} % Set link colors

\newlength{\datebox}\settowidth{\datebox}{Spring 2011} % Set the width of the date box in each block

\newcommand{\NewEntry}[3]{\noindent\hangindent=2em\hangafter=0 \parbox{\datebox}{\small \textit{#1}}\hspace{1.5em} #2 #3 % Define a command for each new block - change spacing and font sizes here: #1 is the left margin, #2 is the italic date field and #3 is the position/employer/location field
\vspace{0.5em}} % Add some white space after each new entry

\newcommand{\Description}[1]{\hangindent=2em\hangafter=0\noindent\raggedright\footnotesize{#1}\par\normalsize\vspace{1em}} % Define a command for descriptions of each entry - change spacing and font sizes here
\newcommand{\MoreDescription}[1]{\hangindent=2em\hangafter=0\noindent\raggedright\scriptsize{#1}\par\normalsize\vspace{1em}} % Define a command for descriptions of each entry - change spacing and font sizes here

%----------------------------------------------------------------------------------------

\begin{document}

\thispagestyle{empty} % Stop the page count at the bottom of the first page

%----------------------------------------------------------------------------------------
%	NAME AND CONTACT INFORMATION SECTION
%----------------------------------------------------------------------------------------

\begin{cv}{Tim {\Large CD} Lucas}\vspace{1.5em} % Your name

\noindent{\color{Maroon}\spacedlowsmallcaps{Personal Information}}\vspace{0.5em} % Personal information heading

%\NewEntry{}{\textit{Born in Canada,}}{05 September 1987} % Birthplace and date

\NewEntry{email}{\includegraphics[height=8pt]{Ar_Icon_Contact} \href{mailto:timcdlucas@gmail.com}{timcdlucas{\small @}gmail.com}} % Email address

\NewEntry{twitter}{\includegraphics[height=8pt]{Twitter_logo_blue-small.png} \href{http://www.twitter.com/timcdlucas}{{\small @}timcdlucas}  \href{http://www.twitter.com/statsforbios}{{\small @}statsforbios}} % Personal website

\NewEntry{website}{\href{http://www.ucl.ac.uk/~ucbptcl}{www.ucl.ac.uk/{\raise.17ex\hbox{$\scriptstyle\mathtt{\sim}$}}ucbptcl}} % Personal website

\NewEntry{github}{\href{http://www.github.com/timcdlucas}{www.github.com/timcdlucas}} % Personal website

\NewEntry{scholar}{\href{http://scholar.google.co.uk/citations?user=WfpSfMAAAAAJ&hl=en&oi=ao}{Google scholar}} % Personal website



\NewEntry{phone}{07415 863 536} % Phone number(s)

\vspace{1em} % Extra white space between the personal information section and goal

\hspace{-0.4cm}{\color{Maroon}\spacedlowsmallcaps{Present Appointment}}\vspace{1em}

\NewEntry{2016--Present}{University of Oxford, Malaria Atlas Project}

\Description{\MarginText{Post Doc.}My current position is as a postdoctoral research scientist in geospatial epidemiology with the \href{www.seeg.well.ox.ac.uk}{Malaria Atlas Project} at the University of Oxford. I use spatial statistics and machine learning methods to map infectious and vector borne diseases.}

\hspace{-0.4cm}{\color{Maroon}\spacedlowsmallcaps{Previous Appointments}}\vspace{1em}

\NewEntry{Jan--July 2016}{CBER, UCL}

\Description{\MarginText{Research Programmer}I was the staff programmer for the Centre of Biodiversity and Environment Research at UCL. I worked on two main projects. I worked with the \href{www.madingleymodel.org/}{Madingley Model}---an ecological model of all life, written in C\#  to enable this model to run on the {\cemph high performance cluster} at UCL. Secondly, I translated code from {\cemph Mathematica to R} for analyses of measurements of 3D objects used in {\cemph paleontological research}. I also provide technical support for the rest of the department.}

%------------------------------------------------

\vspace{1em} % Extra space between major sections


%----------------------------------------------------------------------------------------
%	EDUCATION
%----------------------------------------------------------------------------------------

{\color{Maroon}\spacedlowsmallcaps{Education}}\vspace{1em}

\NewEntry{2012--2016}{University College London, {\small CoMPLEX}}

\Description{\MarginText{PhD}\textit{The role of population structure and size in determining bat pathogen richness}\newline
Description: I used {\cemph network models} and comparative methods to study the epidemiology of bat-borne diseases. As bats carry a number of important {\cemph zoonotic diseases}, understanding the spread of these diseases within the bat population and how this affects spillover to humans and livestock is increasingly important. The unusually social nature of bat populations will strongly affect how diseases spread.\newline
Supervisors: \textsc{Prof.~Kate Jones} \& \textsc{Dr~Hilde Wilkinson-Herbots}}

%------------------------------------------------

\NewEntry{2011--2012}{University College London, {\small CoMPLEX}}

\Description{\MarginText{MRes}\textit{Modelling Biological Complexity} $\cdotp$ {\bf Merit}\newline
Description: Projects included adapting `{\cemph ideal gas}' models to acoustic data, analysing moment closures for a {\cemph pair-approximation} model of plant ecology and applying a novel {\cemph machine learning} method to a library of bat calls. }


%\NewEntry{Summer 2012}{Estimating abundances using acoustic data}

%\Description{\MarginText{Summer Project}I adapted `{\cemph ideal gas}' models to acoustic data. I applied the model using R to a pan-European bat survey. We have worked on this project further, validating results with simulations, and the work is now published.}

%------------------------------------------------

%\NewEntry{May 2012}{Pair approximations in spatial biology}

%\Description{\MarginText{Case Presentation}I compared a number of moment closures for a {\cemph pair-approximation} model of tree population growth to lattice simulations written in Mathematica. [\href{http://files.figshare.com/92192/Pair_approximations_in_spatial_biology_Tim_Lucas_2012.pdf}{pdf}]}


%------------------------------------------------

%\NewEntry{Dec. 2011}{Gaussian processes for bat identification}

%\Description{\MarginText{Case Presentation}I applied a novel {\cemph machine learning} method to a library of {\cemph bat calls} in Matlab. I compared the effectiveness of this method to standard machine learning methods applied in R. [\href{http://www.ucl.ac.uk/~ucbptcl/bat-gaussian-process-learning-tim-lucas-2011.pdf}{pdf}]}


%------------------------------------------------

\NewEntry{2006--2010}{University of Sheffield, {\small Animal \& Plant Sciences}}

\Description{\MarginText{MBioSci}\textit{Zoology} $\cdotp$ {\bf First}\newline
Description: For my final project I used {\cemph wavelet} analysis to study multi-annual {\cemph cycles in malaria} incidence in Thailand.}

%------------------------------------------------
\vspace{1em} % Extra space between major sections


%----------------------------------------------------------------------------------------
%	WORK EXPERIENCE
%----------------------------------------------------------------------------------------

\noindent{\color{Maroon}\spacedlowsmallcaps{Previous Appointments}}\vspace{1em}




%------------------------------------------------

\NewEntry{Autumn 2014}{Zo\"{o}n: An R package for reproducible SDMs}

\Description{\MarginText{Internship}I wrote the first version of an R package for {\cemph reproducible} species distribution modelling. The package uses an online repository of user submitted `modules' to allow the software to keep up with this fast moving field and allow analyses to be completely reproducible. [\href{http://www.github.com/zoonproject/zoon}{Github}]}

%------------------------------------------------

\NewEntry{August 2011}{Smithsonian Tropical Research Institute}

\Description{\MarginText{Volunteer Fieldwork}Two months fieldwork in Panam\'{a} on two projects: studying \emph{Anolis} dewlap evolution and studying gut length plasticity in Red-eyed tree frogs. }

%------------------------------------------------

\NewEntry{May 2011}{Chilo\'{e} Silvestre, Chil\'{e}}

\Description{\MarginText{Volunteer Fieldwork}I spent two weeks trapping Darwin's foxes in Chil\'{e} to collect samples for geophylogenetics.}

%------------------------------------------------

\NewEntry{August 2010}{University of Sheffield}

\Description{\MarginText{Summer Internship}I studied the evolutionary response of plant communities to climate change with Dr Raj Whitlock. I collected, propagated and analysed plants collected from the field. }

%------------------------------------------------

\NewEntry{August 2009}{University of York, {\small YCCSA}}

\Description{\MarginText{{\footnotesize TRANSIT} Internship}I studied collective foraging behaviour by programming a {\cemph complex 3{\footnotesize D} foraging} {\cemph model} in Java and running simulations on a cluster at the York Centre for Complex Systems Analysis.}

%------------------------------------------------

\vspace{2em} % Extra space between major sections
{\color{Maroon}\spacedlowsmallcaps{Other Appointments and Affiliations}}\vspace{1em}

\Description{\MarginText{Peer Review}\emph{Journals Reviewed for:}}

\vspace{-0.5em} % Negative vertical space to counteract the vertical space between every \Description command

\Description{\ \ $\cdotp$\ \ Methods in Ecology and Evolution, National Academy Science Letters}

%\Description{\MarginText{Society Membership}2014--present\ \ $\cdotp$\ \ Royal Society of Tropical Medicine and Hygiene: Student member}


%------------------------------------------------

\vspace{2em} % Extra space between major sections

%----------------------------------------------------------------------------------------
%	OTHER INFORMATION
%----------------------------------------------------------------------------------------

{\color{Maroon}\spacedlowsmallcaps{Conferences}}\vspace{1em}

\Description{\MarginText{2016}\emph{Population structure and pathogen richness in bats.}}\vspace{-1em}
\MoreDescription{Presentation at \href{http://www.epidemics.elsevier.com/previous-conferences.asp}{Epidemics5}, Elsevier by \textbf{Lucas TCD}}
\vspace{-0.5em}

\Description{\emph{Using gas models to model camera trap and acoustic sensor surveys.}}\vspace{-1em}
\MoreDescription{Presentation at \href{https://www.kent.ac.uk/smsas/statistics/research/seak-news.html?view=232}{Statistical Ecology Research Festival}, University of Kent by \textbf{Lucas TCD}}
\vspace{-0.5em}

\Description{\MarginText{2015}\emph{The Zo\"{o}n Project: Reproducible, Remixable and Shareable Species Distribution Modelling with R.}}\vspace{-1em}
\MoreDescription{Presentation at \href{http://www.britishecologicalsociety.org/events/current_future_meetings/past-bes-annual-meetings/2015-annual-meeting/}{BES Annual Meeting} by August T, Golding N, \textbf{Lucas TCD}, Gavaghan D, Isaac N, O'Hara B, van Loon E \& McInerny G }

\vspace{-0.5em}


\Description{\emph{Simple, Shareable and Reproducible Species Distribution Modelling with the Zo\"{o}n R package.}}\vspace{-1em}
\MoreDescription{Poster at \href{http://www.britishecologicalsociety.org/events/current_future_meetings/past-bes-annual-meetings/2015-annual-meeting/}{BES Annual Meeting} by N Golding, \textbf{Lucas TCD}, August T, Gavaghan D, Isaac N, O'Hara B, van Loon E \& McInerny G}

\vspace{-0.5em}

\Description{\emph{Comparative and computational studies of pathogen richness in bats.}}\vspace{-1em}
\MoreDescription{Presentation at Research in Progress, \href{https://rstmh.org/events/research-progress-2015}{RSTMH} by \textbf{Lucas TCD}, Wilkinson-Herbots H \& Jones KE.}

\vspace{-0.5em}

\Description{\emph{A comparative and computational study of population structure and pathogen richness in bats.}}\vspace{-1em}
\MoreDescription{Presentation at \href{http://www.epidemics.elsevier.com/}{Epidemics5} conference by \textbf{Lucas TCD}, Wilkinson-Herbots H \& Jones KE.}

\vspace{-0.5em}



\Description{\emph{An ideal gas model for estimating absolute abundances from bat detector data.}}\vspace{-1em}
\MoreDescription{Presentation at the National Bat Conference. [\href{http://www.slideshare.net/timcdlucas/tim-lucasnbc}{slides}]}

\vspace{-0.5em}

\Description{\emph{Pathogen diversity and bat population structure.}}\vspace{-1em}
\MoreDescription{Poster at British Parasitological Society Autumn Meeting.}


\vspace{-0.5em}

\Description{\emph{Estimating abundance from camera traps and acoustic sensors.}}\vspace{-1em}
\MoreDescription{Presentation at {\footnotesize CEH}, Wallingford seminar series.}


\vspace{-0.5em}

\Description{\MarginText{2014}Presentation at \href{http://id2-ox.co.uk/}{id2oxford} conference. [\href{http://www.slideshare.net/timcdlucas/tim-lucasid2ox}{slides}]}
\vspace{-0.5em}

\Description{Poster at the {\footnotesize CoMPLEX} conference. [\href{http://figshare.com/articles/Estimating_abundance_from_camera_traps_and_acoustic_sensors/1321269}{pdf}]}
\vspace{-0.5em}

\Description{\MarginText{2013}Presentation at BritBats 2 [\href{http://www.slideshare.net/timcdlucas/tim-lucasbritbats}{slides}].}

\vspace{-0.5em} % Negative vertical space to counteract the vertical space between every \Description command

\Description{Invited attendance at \href{http://ecoviz.wordpress.com/1-joining-things-up-and-filling-the-gaps/}{eco{\footnotesize VIZ}} Tansley workshop.}

\vspace{-0.5em} % Negative vertical space to counteract the vertical space between every \Description command

\Description{Poster at the {\footnotesize CoMPLEX} conference and id2 conference. [\href{http://figshare.com/articles/The_Phylogeography_of_Henipah_Virus/1046697}{pdf}]}


\vspace{1em} % Extra space between major sections

%----------------------------------------------------------------------------------------
%	PUBLICATIONS
%----------------------------------------------------------------------------------------
{\color{Maroon}\spacedlowsmallcaps{Publications}}\vspace{1em}

\Description{\MarginText{2017}Redding D, {\bf Lucas TCD}, Heath A and Jones KE. \emph{Evaluating Bayesian spatial methods for modelling species distributions models with clumped and restricted data.} Submitted}

\Description{{\bf Lucas TCD}, Herbots HM, \& Jones KE. \emph{A mechanistic model to compare the importance of interrelated population measures on pathogen richness: host population size, density and colony size.} In prep.}

\Description{Curnick DJ, Koldewey HJ, {\bf Lucas TCD}, Jones KE \& Collen B. \emph{Detecting changes in pelagic shark populations using remote cameras.} In review.}

\Description{\MarginText{2015}{\bf Lucas TCD}\textsuperscript{$\ast$}, Moorcroft EA\textsuperscript{$\ast$}, Freeman R, Rowcliffe MJ \& Jones KE. (2015) \emph{A generalised random encounter model for estimating animal density with remote sensor data.} Methods in Ecology and Evolution. doi: 10.1111/2041-210X.12346 [\href{http://onlinelibrary.wiley.com/doi/10.1111/2041-210X.12346/epdf}{pdf}]}

\Description{\MarginText{2013}Walters CL, Collen A, {\bf Lucas TCD}, Mroz K, Sayer CA and Jones KE. (2013) Challenges of Using Bioacoustics to Globally Monitor Bats. in \emph{Bat Evolution, Ecology, and Conservation.} Springer New York. 479-499.}

\Description{\MarginText{} \scriptsize{ $\ast$ Co-first authors. } }

%------------------------------------------------

\vspace{1em} % Extra space between major sections


%----------------------------------------------------------------------------------------
%	Software
%----------------------------------------------------------------------------------------

{\color{Maroon}\spacedlowsmallcaps{Software}}\vspace{1em}

\Description{\MarginText{On CRAN}Goswami A, {\bf Lucas TCD}, Sivasubramaniam P, Finarelli J (2016) \emph{A Maximum Likelihood Approach to the Analysis of Modularity}. \url{www.github.com/timcdlucas/EMMLi}}

\Description{{\bf Lucas TCD}, Goswami A (2016) \emph{paleomorph: Geometric Morphometric Tools for Paleobiology}. \url{www.github.com/timcdlucas/paleomorph}}

\Description{{\bf Lucas TCD}, Golding N, August T, McInerny G, van Loon E (2015) \emph{Zo\"{o}n: Reproducible, Accessible \& Shareable Species Distribution Modelling}. \url{www.github.com/zoonproject/zoon}}

\Description{{\bf Lucas TCD} (2015) \emph{palettetown: Use Pokemon Inspired Colour Palettes} \url{www.github.com/timcdlucas/palettetown}}


\vspace{1em} % Extra space between major sections

{\color{Maroon}\spacedlowsmallcaps{Teaching}}\vspace{1em}

\Description{2015\ \ $\cdotp$\ \ Demonstrator for reproducible species distribution modelling workshop run by Quantitative Ecology special interest group at BES.}
\vspace{-0.5em} % Negative vertical space to counteract the vertical space between every \Description command

\Description{2013--2014\ \ $\cdotp$\ \ Online tutor for \href{http://sysmic.ac.uk/home.html}{{\footnotesize SysMIC}}, a course for teaching quantitative skills to biologists.}

\vspace{1em} % Extra space between major sections


%----------------------------------------------------------------------------------------
%	COMPUTER SKILLS
%----------------------------------------------------------------------------------------

{\color{Maroon}\spacedlowsmallcaps{Computational Skills}}\vspace{1em}

\Description{\MarginText{Statistical methods}Geospatial statistics, machine learning, Bayesian inference.}

\Description{\MarginText{Programming Languages}R (eight years), Python, Matlab, Mathematica, Java, SQL.}

\Description{\MarginText{OS}Comfortable with Windows, Mac or Linux.}

\Description{\MarginText{Other}Experience in R package development, Git/Github, unit testing, LaTeX, web design, markdown, shell/ssh and high performance computing.}



\vspace{1em} % Extra space between major sections


%----------------------------------------------------------------------------------------
%	COMPUTER SKILLS
%----------------------------------------------------------------------------------------

{\color{Maroon}\spacedlowsmallcaps{Referees}}\vspace{-1em}

\begin{multicols}{2}
\begin{footnotesize}
\textsc{Prof. Kate Jones}\newline
Chair of Ecology and Biodiversity\newline
Centre for Biodiversity and Environment Research\newline
University College London\newline
Gower Street\newline
London\newline
United Kingdom\newline
{\scriptsize WC}1{\scriptsize E} \ 6{\scriptsize BT}

\vfill
\columnbreak
\textsc{Dr Greg McInerny}\newline
Centre for Interdisciplinary Methodologies\newline
University of Warwick\newline
Coventry\newline
United Kingdom\newline
{\scriptsize CV}4 \ {\scriptsize 7AL}

\end{footnotesize}
\end{multicols}
\vspace{-0.5cm}
\begin{multicols}{2}
\begin{footnotesize}
Email: \href{mailto:kate.e.jones@ucl.ac.uk}{kate.e.jones{\scriptsize @}ucl.ac.uk}\newline
Tel: +44 (0)20 31084230
\columnbreak

Email: \href{mailto:g.mcinerny@warwick.ac.uk}{G.McInerny{\scriptsize @}warwick.ac.uk}\vspace{-0.8mm}\newline
Tel: +44 (0)2476 574710
\end{footnotesize}
\end{multicols}




\end{cv}

\end{document}
